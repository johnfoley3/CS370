\documentclass[11pt]{article}

\usepackage{setspace}
\usepackage[letterpaper,margin=1in]{geometry}
\usepackage[parfill]{parskip}
\newcommand{\tab}{\hspace*{2em}}

\title{Final Exercise}
\author{John Foley}
\date{6 April 2014}

\begin{document}
\maketitle

\thispagestyle{empty}

\begin{doublespace}

\emph{Question 1}

\tab I will be talking about my two teams separately here. First, I will discuss my small group team before the merge, and then the subgroup that came together after the merge. The first group consisted of myself, Caitlyn, Daniel, Bryce, Josh, and Cody. The second group I worked with consisted of myself, Brandon, Anna, and Nick. 

\tab The first group worked primarily from the initiation of dialogue with the client to user stories. Our main purpose was to elicit domain details, requirements, and subsequent user stories from predetermined questions. Our closest way of matching proper software engineering practices and techniques was in the form of thoroughness and traceability. I feel that we kept track of details about questions well enough to form strong requirements, which the entire system was modeled after. We were very new to the application domain, and attempted to compensate by taking notes, discussing wording of questions in length amongst ourselves, and clearly tracing each phase to the previous phase. This way we could remember our reasoning, and adapt the project all the way to the foundation if necessary.

\tab The second group worked primarily from the creation of user stories to the end of the application life as we know it. We worked very well together, and I'd say the one way that the team matched proper sotware engineering practices most was using Test Driven Development to shape the end product for a user story. We would either pick a user story for the iteration, or be given it by another group, and then immediately talk it through and create an appropriate test. This test was built and given values that most closely matched what the user story was meant to achieve. This guided us in the actual implementation and helped us to realize the power and all knowing consistency of the green bar. 

\emph{Question 2}

\emph{Question 3}

\emph{Question 4}

\emph{Question 5}


\end{doublespace}


\end{document}