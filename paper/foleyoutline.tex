\documentclass[11pt]{article}

\usepackage[letterpaper,margin=1in]{geometry}

\title{Agile Development Methodology and Its Advantages}
\author{John Foley}
\date{13 April 2014}
\begin{document}
\maketitle

\thispagestyle{empty}

\begin{abstract}
  Agile Methodologies have taken the software engineering industry by storm, especially in this modern
era of internet technology and application production. The Agile Manfiesto, published in 2001,
 introduced many terms and ideas to describe the current set of methodologies that we call Agile. There are many
 problems in software engineering that Agile Methodologies are designed to address, such as ability to react to 
 change, maintain communication with clients, and evolutionary development cycles. These problems are inherent
 in software development, and are not handled well in past methodologies such as Waterfall. 
\end{abstract}

\section{Iterative Cycles}

One strength of the Agile Methodology is how the software development life cycle is split up. Agile thinks
of the time it takes to solve a specific feature or problem as an iteration, typically two weeks long, but
is up to the production manager. This allows for structured development of the application, and breaks
up the requirements from the user into smaller, easier to handle chunks. 

\subsection{Iterations} 
The structured phases of feature and problem development of the application

\subsection{Separation into Easier Phases} 
Each iteration allows for the application to be developed in parts. This allows for flexible changes and
extremely reactive development.~\cite{Cao:2010:MDA:1877725.1877730}

\section{Small, Cross-functional Teams}

\subsection{Small Teams and Scrums}
Teams are kept small in order to maintain inter-team communication. Internal workings are easier to maintain and manage,
as well as becoming more comfortable with whom a developer works with.

\subsection{Scrums and Meetings}
Scrums are quick, efficient meetings for
teams to keep on track. Scrums are held often and typically report driven.

\section{Evolutionary Development}

\subsection{Test Driven Development}
The use of Test Driven Development allows for an application that ensures that past work done on the application
is still working with the addition of new code.

\subsection{Increment} 
Each iteration has a specific goal in mind. After each iteration is complete, the application is one
step further, and considered incremented forward.

\section{Past Methodologies}
The Waterfall Methodology has been the primary methodology used for software development for decades, and is still used today.
These methodologies are effective, but do not address major problems that lead to either failure of a software development
project or shortcomings of an application.

\subsection{Phases Slow Reaction Time}
Waterfall is typically held with several phases, one of which is the design phase, followed by implementation
phase. These phases are essentially "closed door" meetings for developers from clients, and so divergence and thus
failure to respond to change is a problem.

\subsection{Maintain Communication}
Waterfall is not known to maintain strong client communication, and without constant updates and checks on progress,
the project will inevitably diverge from what the client is asking for.

\subsection{Evolution of the Product}
Waterfall methodologies advises development to be carried out in phases, without clear benchmarks to be made. A common problem
is that the project is developed all at once and changing subcomponents becomes difficult and overwhelms a project.

\nocite{*}

\bibliographystyle{plain}
\bibliography{master}

\end{document}